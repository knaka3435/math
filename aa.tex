\documentclass{article}
\usepackage[utf8]{inputenc}
\usepackage{lindrew}[formal]
\usepackage{import}
\title{Read Math}
\author{Kenji Nakagawa}
\date{July 2019}

%Hacky shit
\usepackage{enumitem}

\iffalse
\def\exc{1}
\if\exc1
    \includeonly{C1}
\else
    \includeonly{}
\fi
\fi


\begin{document}

\maketitle
\iffalse
\newenvironment{Exc}[1]
    {\if\exc1
    #1
    \else
    asdf
    \fi
    \
    }
    {
    \
    }
\fi

\section*{Introduction}
Notes are based on the second edition of Number Fields by Daniel A. Marcus, and are meant for private usage only.


\section{Fermat's Conjecture}
Fermat's last theorem has proved to be elusive, and to have inspired a great deal of algebraic number theory. It states that there does not exist nontrivial solutions to \(x^n+y^n=z^n \) whenever \(n > 2\). Kummer saw that the left hand side factored into \(\prod_{i=1}^p (x+\zeta_p^iy)\), and naturally, wanted to show that \(\ZZ[\zeta_p] \) had a unique factorization, and consequently, that the case in which \( p\) does not divide \(x,y\), or\(z\). However, this is only true for regular primes.
\begin{definition}
A prime is \textit{regular} if \(p \nmid h \), where \(h\) denotes the class number of the ring \(\ZZ[\zeta]\), which is the number of equivalence classes of the ideals, with equivalence between ideals \(A,B\) being defined \(A \sim B\) iff \(aA=bB, \ a \in A, b \in B\).
\end{definition}

It can be shown that if \(p \nmid x,y,z \), then there are no solutions given that \(\ZZ[\zeta]\) is a unique factorization domain which implies that \(x+y\zeta \) has the form \(u\alpha^p\) for some \( \alpha \in \ZZ[\zeta]\)
For the following problems, Define \(N: \ZZ[i] \to \ZZ \) by \(N(a+bi) = a^2+b^2\)
\begin{problem}
Verify that for all \(\alpha, \beta \in \ZZ[i], N(\alpha\beta)=N(\alpha)N(\beta) \), either by direct computation or by using the fact that \(N(a+bi)=(a+bi)(a-bi) \). Conclude that if \(\alpha | \gamma \) in \( \ZZ[i]\), then \(N(\alpha) | N(\gamma) \) in \(\ZZ\).
\end{problem}
Suppose \( \gamma = \alpha \beta \), then we have that \(N(\alpha)N(\beta) = N(\alpha\beta) = N(\gamma) \), from which the conclusion is obvious since the sum of two integer squares is yet another integer.
\begin{problem}
Show that every nonzero, non-unit Gaussian integer \( \alpha \) is a product of irreducible elements, by induction on \(N(\alpha)\)
\end{problem}
Suppose \(\alpha\) is irreducible. Then, we can write \(\alpha = \alpha \cdot 1 \). Now, suppose \( \alpha \) is reducible, that is, there exists non-units \(\beta, \gamma\) such that \( \alpha = \beta \gamma \), and thus \( N(\alpha) = N(\beta) N(\gamma) \). Suppose we have proved it for all elements whose norm is less than that of \( \alpha \), then, since \(\beta, \gamma\) are both non-units, their norm is greater than one, and so the conclusion follows. Also, it's apparent that since \(2\) is prime in \(\ZZ\), the elements whose norm is \(2\) are irreducible. Thus we have our base case established as well as our inductive step, therefore we have proved it for all \(\alpha\).
\begin{problem}
Show that \(\ZZ[i]\) is a principal ideal domain (PID); i.e., every ideal \(I\) is principal (this implies that \(\ZZ[i]\) is a UFD).
\end{problem}
This is a common approach: Suppose \(a\) is an element of minimal norm of the ideal \(I\). Then, the ideal generated by \(a\) is going to be contained in \(I\), and on \(\CC\) it takes the shape of a lattice of squares. Now, suppose that there is an element, \(b\) that is not on the lattice. Since \(b \in I \implies ub-ka \in I, \ k \in \ZZ[i] \), where \(u\) is some unit. So, we can obtain \(b'\) that lies in the square whose vertices are \(0, \ a, \ ia, \ a+ia \). Now, we have two cases: the norm of \(b'\) is less than that of \(a\), in which case that's a contradiction, and the other in which the norm of \(b'\) is greater than or equal to that of \(a\). However, when we subtract \(a+ia\) from \(b\), we end up in the portion of the square whose vertices are \(0 \ -ia \ -a, \ -a-ia \) which is closer to the origin than before, another contradiction. Thus, \(\ZZ[i]\) is a principal ideal domain.
\begin{problem}
Use unique factorization in \(\ZZ[i]\) to prove that every prime \(p \equiv 1 \mod 4\) is a sum of two squares.
\end{problem}
It's a well known fact that \(\ZZ_p^*\) has a generator, say \(g\). Using this fact, one can prove that \(g^{\frac{p-1}{2}} \equiv -1 \mod p\) using basic number theory. Since \(p \equiv 1 \mod 4\), we have that \( \frac{p-1}{4} \) is an integer, and thus that \(g^{\frac{p-1}{4}}\) is precisely that \(n\) such that \(n^2 \equiv -1 \mod p\).

Observe that \(p|n^2+1=(n+i)(n-i)\), so evidently, \(p\) cannot be irreducible due to unique factorization, so we can write \(p=ab, \ a,b \in \ZZ[i] \) where neither is a unit. Since \(N(p)=p^2=N(a)N(b)\), we can conclude the desired statement.
We can describe all irreducibles in \(\mathbb{Z}[i]\) as elements whose norm is \(p\) or \(p^2\), where in the second, \( p \equiv 3 \mod 4\).

\begin{problem}
Show that \(\ZZ[\zeta_3]\) is yet another UFD.
\end{problem}

Observe that in this case, we define the norm in the prototypical way, the product of the images of the number in all of the different embeddings, thus \(N(a+b\zeta_3)=(a+b\zeta_3)(a+b\zeta_3^2)\). By a similar argument, we can reduce our lattice on \(\mathbb{C}\) into a parallelogram whose vertices are \(a, a\zeta_3, 0, a+a\zeta_3\), and so the point \(b\) which we assumed was not on the lattice will fall either in a place whose norm is less than that of \(a\), which contradicts the normality, or, we can translate it into a region (which is opposite the original), which has it in an area where the norm is less than that of \(a\). Thus \(\ZZ[\zeta_3]\) is also a UFD. (A side note is that when \(a \in \ZZ[\zeta_3]\) is written as a sum of real and imaginary components, the norm is the sum of the square of each ones magnitude.

\begin{problem}
a
\end{problem}


\subsection{Number Fields and Number Rings}
\begin{definition}
A \textit{number field} is a subfield of \(\CC\) having finite degree (dimension as a vector space) over \(\QQ\). We refer to \(Q[\zeta]\), \(Q[\sqrt{m}]\) as cyclotomic fields, and quadratic fields, with the latter having a subdistinction of real versus imaginary.
\end{definition}
As a result of the following equivalences:
\begin{enumerate}
    \item \(\alpha\) is an algebraic integer
    \item The additive group is finitely generated
    \item \(\alpha\) is a member of some subring of \(\CC\) having a finitely generated additive group
    \item \(\alpha A \subset A \) for some finitely generated additive subgroup \(A \subset \CC\)
\end{enumerate}
we can show that the sum and product of two algebraic numbers is yet another algebraic number. As a more general result, the ring of algebraic integers, notated as \(\mathbb{A} \) in this text or, more commonly, as \(\mathcal{O}\). Furthermore, we may denote the algebraic numbers over some ring as \(\mathbb{A} \cup K\) or as \(\mathcal{O}_K\).

Something that is rather intuitive, but is of note is that \(\gal{\QQ[\zeta_m]}{\QQ} \simeq \left(\ZZ/m\ZZ\right)^{\times}\). This implies that the Galois group is transitive, and furthermore, that the subfields correspond to the subgroups, as well as that when \(m=p\), a prime, that it contains a unique subfield of each degree dividing \(p-1\), and that there it contains a unique quadratic field, which are \(\sqrt{\pm p}\) depending on \(p\).

Sometimes it's useful to replace embeddings of a number field with automorphisms. One way to do his is to extend the number field to a normal extension of \(\ZZ\), which is always possible. As a reminder, a normal extension is an extension that any irreducible polynomial in the original field is either still irreducible, or it factors completely into linear terms. As an example, \(L=\QQ[\sqrt[3]{2},\zeta_3]\) is a normal extension of \(\QQ[\sqrt[3]{2}]\).

\begin{definition}
We define the trace and the norm as \( T(\alpha) = \sum_{i}^n \sigma_i(\alpha) \) and \(N(\alpha) = \prod_{i}^n \sigma_i(\alpha)\), where \(\sigma_i(\alpha)\) is the image of \(\alpha\) in the \(i\)th embedding. 
\end{definition}
It is easy to show that both are always rational based on the minimal polynomial, and to show that the norm of an algebraic integer is an integer.

Some applications of this is that in general (except for two cases), that the imaginary integer fields have the property that a number is a unit iff it has norm \(\pm1\).

\begin{definition}
The relative trace and relative norm is essentially the same, \(T^L_K, N^L_K\) are the embeddings of \(L\) in \(\CC\) which keep \(K\) fixed.
\end{definition}
Similarly to field extensions, the relative trace, and relative norm are also transitive.

Additionally, we define the discriminant of an \(n\)-tuple as \(\disc(\alpha_1, \dots, \alpha_n = |\sigma_i(\alpha_j)|^2\), where \(\sigma_i\) are the embeddings of the field that \(\alpha_i\) are in, where \(|a_{ij}|\) represents the determinant of the matrix \([a_{ij}]\), which has \(a_{ij}\) in the standard position. By some linear algebra stuff, we can define \(\disc(\alpha_1, \dots \alpha_n)=|T(\alpha_i\alpha_j)|\). 
This has some nice properties which include that if the argument are all algebraic integers, then it is an integer, and if the determinant is nonzero, then the arguments are linearly independent over the rationals.

\import{}{C2.tex}
\end{document}
