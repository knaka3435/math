\begin{problem}
Show that every number field of degree \(2\) over \(\QQ\) is one of the quadratic fields \(\QQ[\sqrt{m}, m \in \ZZ\). Show that the fields \(\QQ[\sqrt{m}], m\) squarefree, are pairwise distinct. 
\end{problem}
Observe that any field of degree \(2\) over \(\QQ\) is going to be a quadratic field, as \(2\) is prime, and hence the only extensions possible are roots of nontrivial quadratic equations. Without loss of generality, we can assume that the root is the square root of a rational number. Furthermore, since it is a field, we are able to scale the root by any integer which allows us to change it from the square root of a rational number to the square root of some integer since they are isomorphic field extensions. As for the second part, Suppose \(\QQ(\sqrt{n}) \sim \QQ(\sqrt{m}), n \ne m\). Then \(\exists \phi: \QQ(\sqrt{n}) \to \QQ(\sqrt{m})\). Evidently, this must fix \(0,1 \implies a,\ a \in \ZZ \implies \frac{a}{b},\ a,b \in \ZZ^{\times}\), in other words, all of \(\QQ\). Now consider \(\phi(n)=\phi(\sqrt{n}^2)=\phi(\sqrt{n})^2 = n\). However, \(n\) is squarefree, thus \(\sqrt{n}=\sqrt{m}\).
\begin{problem}
Let \(I\) be the ideal generated by \(2\) and \(1+\sqrt{-3}\) in the ring \(\ZZ[\sqrt{-3}]\). Show that \(I \ne (2)\) but \(I^2=2I\). Conclude that ideals in \(\ZZ[\sqrt{-3}]\) do not factor uniquely into prime ideals. Show moreover that \(I\) is the unique prime ideal containing \((2)\) and conclude that \((2)\) is not a product of prime ideals.
\end{problem}
We compute the product of two elements in \(I, \ (2a+b+b\sqrt{-3})(2c+d+d\sqrt{-3}=4ac+2(bc+bc\sqrt{-3})+2(ad+ad\sqrt{-3})+2(-bd+bd\sqrt{3})\). Observe that when \(d=0, c = 1\) that \( 2I \subset I^2\) and that when we subtract \(4ac+2(bc+ad)(1+\sqrt{-3}) \in 2I \) from the product, we get \(-2bd+2bd\sqrt{-3} = 2bd(1+\sqrt{-3})-4bd \in 2I\). Thus \(I^2 \subset 2I \implies I^2=2I\).

Now, the next part is rather tricky. By considering norms, it becomes obvious that if a prime ideal contains \((2)\), then it must have \(2\) as one of its generators. Thus, we wish to seek all prime ideals of the form \(I=(2,a+b\sqrt{-3})\) and wish to show that the only one that exists has a odd and \(b=1\) (this becomes \(I\)). Although this is not a Dedekind domain, We can use the fact that \(\ZZ[\sqrt{-3}]\) is a free abelian group of rank \(2\) to show that the ideals are free abelian groups of rank at most \(2\). Suppose \(a\) is even. Then this ideal is equivalent to \((2,b\sqrt{-3})\) which is equal to \((2,\sqrt{-3})\) if \(b\) is odd (\(b\) even gives \((2)\), which is obviously not prime). Observe that \((1+\sqrt{-3})^2=-2+2\sqrt{-3}\) which implies that it is not prime. Now, suppose \(a\) is odd, then it is equivalent to \((2,1+b\sqrt{-3})\). If \(b\) is even, we get \((1)\), which is not prime, thus \(b\) must be odd, and we can reduce it to \((2,1+\sqrt{-3})\). Now, we wish to prove that I is a prime ideal. Consider \(\frac{\ZZ[\sqrt{-3}]}{(2,1+\sqrt{-3})}\). The number \(a+b\sqrt{-3} \in \ZZ[\sqrt{-3}]\) is sent to \(a-b \mod 2\), which is either \(0,1\). This is evidently a field, and hence an integral domain.
\begin{problem}
Let \(m\) be a squarefree integer. The set of algebraic integers in the quadratic field \(\QQ[\sqrt{m}]\) is \(\{a+b\sqrt{m}| a,b \in \ZZ\}\) if \( m \equiv 2,3 \mod 4\) or \(\{\frac{a+b\sqrt{m}}{2} | a,b \in \ZZ, \ a\equiv b \mod 2\}\) if \(m \equiv 1 \mod 4\).
\end{problem}
Let \(\alpha = r+s\sqrt{m},\ r,s \in \QQ\). If \(s \ne 0\), then the minimal polynomial of \(\alpha\) is \(x^2-2rx+r^2-ms^2\). We wish to show that \(2r, r^2-ms^2\) are integers precisely when the above happens. \(2r \in \ZZ \implies r \in \frac{1}{2}\ZZ \implies s \in \frac{1}{2}\ZZ\). Suppose \(r \in \frac{1}{2}\ZZ\) but is not an integer. Then, we require \(4ms^2\) to have a remainder of \(1\). Since \(s^2\) is always equivalent to \(1 \mod 4\), we have that \(m \equiv 1 \mod 4\). On the other hand, if \(r,s \in \ZZ\) and \(m=1\), then those also generate algebraic integers. We have shown earlier that if \(m=2,3\), then \(r,s \ZZ\), so we're done.

\begin{problem}
Suppose \(a_0,\dots,a_{n-1}\) are algebraic integers and \(\alpha\) is a complex number satisfying \(\alpha^n+a_{n-1}+\dots + a_1\alpha +a_0 =0\) Show that the ring \(\ZZ[a_0,\dots,a_{n-1},\alpha]\) has a finitely generated additive group. Conclude that \(\alpha\) is an algebraic integer.
\end{problem}
Since \(a_i\) are algebraic, any instance of \(a_i\) whose power is the degree of its minimal polynomial can be replaced with lower order terms, and similarly, we can reduce powers of \(\alpha\), so we have that \(\ZZ[a_0, \dots, a_{n-1}, \alpha]\) is a finitely generated additive group. THE LATTER PART STILL NEEDS A SOLUTION SOMETHING SOMETHING FREE ABELIAN GROUP

\begin{problem}
Show that the Galois group of \(\QQ[\omega]\) over \(\QQ\) is isomorphic to the multiplicitive group of integers \(\mod m\).
\end{problem}
Evidently, the automorphisms \(\omega \mapsto \omega^k \) where \(\gcd(k,m)=1\) work, and composition holds, as we know that \(\omega^a \mapsto \omega^{ak}\) for all relatively prime \(a\). Then, \(\sigma_2\sigma_1\) is going to take \(\omega \mapsto (\omega^{k_1})^{k_2}\). As a result, this composition is literally the multiplicitive group.

\begin{problem}
Let \(\omega=e^{2\pi i/p}, p \) an odd prime. Show that \(\QQ[\omega]\) contains \(\sqrt{p}\) if \(p \equiv 1 \mod 4\) and \(\sqrt{-p}\) if \(p \equiv -1 \mod 4\). Express \(\sqrt{-3}\) and \(\sqrt{5}\) as polynomials in the appropriate \(\omega\). Show that the \(8^{\text{th}}\) cyclotomic field contains \(\sqrt{2}\) and that in general, \(\QQ[\sqrt{m}]\) is contained in the \(\disc(\AA \cap \QQ[\sqrt{m}])\)th cyclotomic field.
\end{problem}
Observe that due to the original definition of the discriminant, as well as various properties, we have that \(\disc(\omega) = \alpha^2 = \pm p^{p-2}\) where \(\alpha \in \QQ[\omega]\), and the plus sign holds if it is \(1 \mod 4\) and negative otherwise. This implies the desired, as you can simply divide by \(p\) repeatedly to get the result for \(m\) prime. Now, we make the following observations: \(\QQ[\zeta_d] \subset \QQ[\zeta_n]\) for all \(d|n\), \(\disc(\AA \cap \QQ[\sqrt{m}]\) is \(4m\) if \(m \equiv 2, 3 \mod 4\) and \(m\) if \(m \equiv 1 \mod 4\), since \(i \in \QQ[i] \subset \QQ[\zeta_{4p}] \implies \sqrt{p} \in \QQ\zeta_{4p}\), suppose \(p_1, p_2 \equiv 3 \mod 4, \implies \sqrt{-p_1}, \sqrt{-p_2} \in \QQ[\zeta_{p_1p_2}] \implies \sqrt{p_1p_2} \in \QQ[\zeta_{p_1p_2}]\). Since either \(m\) contains a factor of \(p\) or not, we don't need to worry about powers, and we've already resolved the case in which \(m\) contains a factor of \(2\), so we're done (reason by pairing).

\begin{problem}{exercise 10}
Show that if \(m\) is even, \(m|r\), and \(\phi(r) \leq \phi(m) \), then \(r=m\).
\end{problem}
Evidently, \(r \ge m\), so assume that \(r > m, \ m=p_1^{a_1}p_2^{a_2} \dots p_i^{a_i}, \ r = p_1^{a_1+b_1}p_2^{a_2+b_2}\dots p_i^{a_i+b_i}p_j^{b_j} \dots\). Then we have that \(\phi(r)=\phi(m)\cdot p_1^{b_1}p_2^{b_2}\dots (p_j-1)p_j^{b_j-1}\dots \), which is necessarily greater than \(\phi(m)\), since we already know that \(\mu_2(r) \ge 1\). Thus we must have that \(r=m\).
\begin{problem}
\begin{enumerate}[label=(\alph*)]
    \item Suppose all roots of a monic polynomial \(f \in \QQ[x], \ \deg(f)=n\) have absolute value \(1\). Show that the coefficient of \(x^r\) has absolute value \(\le \binom{n}{r}\).
    \item Show that there are only finitely many algebraic integers \(\alpha\) of fixed degree \(n\), all of whose conjugates(including \(\alpha\) have absolute value \(1\).
    \item Show that \(\alpha\) must be a root of \(1\).
\end{enumerate}
\end{problem}
Suppose we have two elements \(\alpha, \beta\) that are complex conjugates of each other. Then, \(\alpha\beta= 1\), but \(|\alpha+\beta| \le 2\), and we have that the \(r\)th coefficient can be written as \((\alpha\beta)\sigma_{r-2}+(\alpha+\beta)\sigma_{r-1}+\sigma_r \le \dbinom{n-2}{r-2}+2\dbinom{n-2}{r-1}+\dbinom{n-2}{r}\), where \(\sigma_k\) is the \(k\)th degree symmetric sum of the other roots (with multiplicity). We're (basically) done and it's bound above by \(\dbinom{n}{r}\). For the second part, we can note that the first part restricts the coefficients of the minimal polynomial to a finite set of integer tuples. Finally, for the last part, \(\alpha\) is from a finite set, and hence the powers of it must be finite, and we must have equality of some two powers, from which we can use the cancellation property to arrive at a root of \(1\).

\begin{problem}
Now we can prove Kummer's lemma on units in the \(p^{\textrm{th}}\) cyclotomic field, as stated before. Let \(\omega = e^{2pi i/p}\), \(p\) and odd prime, and suppose \(u\) is a unit in \(\ZZ[\omega]\).
\begin{enumerate}[label=(\alph*)]
    \item Show that \(u/\overline{u}\) is a root of \(1\). Conclude that \(u/\overline{u} = \pm \omega^k\) for some \(k\).
    \item Show that the \(+\) sign holds.
\end{enumerate}
\end{problem}


\begin{problem}
Show that \(\pm 1\) are the only roots in the ring \(\AA \cap \QQ[\sqrt{m}]\), \(m\) squarefree, \(m<0, m \neq -1, -3\).
\end{problem}
In the two \(\neq\) cases, they are in fact a cyclotomic field. Otherwise, we see that the integral basis is going to be \(\begin{cases} \left(\frac{1}{2}, \frac{\sqrt{m}}{2}\right)  & m \equiv 1 \mod 4 \\ (1, \sqrt{m}) & m \not\equiv 1 \mod 4  \end{cases} \), and we can take norms, giving \( \begin{cases} \frac{a^2}{4} -\frac{mb^2}{4} & m \equiv 1 \mod 4 \\ a^2-b^2m & m \not\equiv 1 \mod 4\end{cases} \)