For the following problems, Define \(N: \ZZ[i] \to \ZZ \) by \(N(a+bi) = a^2+b^2\)
\begin{problem}
Verify that for all \(\alpha, \beta \in \ZZ[i], N(\alpha\beta)=N(\alpha)N(\beta) \), either by direct computation or by using the fact that \(N(a+bi)=(a+bi)(a-bi) \). Conclude that if \(\alpha | \gamma \) in \( \ZZ[i]\), then \(N(\alpha) | N(\gamma) \) in \(\ZZ\).
\end{problem}
Suppose \( \gamma = \alpha \beta \), then we have that \(N(\alpha)N(\beta) = N(\alpha\beta) = N(\gamma) \), from which the conclusion is obvious since the sum of two integer squares is yet another integer.
\begin{problem}
Show that every nonzero, non-unit Gaussian integer \( \alpha \) is a product of irreducible elements, by induction on \(N(\alpha)\)
\end{problem}
Suppose \(\alpha\) is irreducible. Then, we can write \(\alpha = \alpha \cdot 1 \). Now, suppose \( \alpha \) is reducible, that is, there exists non-units \(\beta, \gamma\) such that \( \alpha = \beta \gamma \), and thus \( N(\alpha) = N(\beta) N(\gamma) \). Suppose we have proved it for all elements whose norm is less than that of \( \alpha \), then, since \(\beta, \gamma\) are both non-units, their norm is greater than one, and so the conclusion follows. Also, it's apparent that since \(2\) is prime in \(\ZZ\), the elements whose norm is \(2\) are irreducible. Thus we have our base case established as well as our inductive step, therefore we have proved it for all \(\alpha\).
\begin{problem}
Show that \(\ZZ[i]\) is a principal ideal domain (PID); i.e., every ideal \(I\) is principal (this implies that \(\ZZ[i]\) is a UFD).
\end{problem}
This is a common approach: Suppose \(a\) is an element of minimal norm of the ideal \(I\). Then, the ideal generated by \(a\) is going to be contained in \(I\), and on \(\CC\) it takes the shape of a lattice of squares. Now, suppose that there is an element, \(b\) that is not on the lattice. Since \(b \in I \implies ub-ka \in I, \ k \in \ZZ[i] \), where \(u\) is some unit. So, we can obtain \(b'\) that lies in the square whose vertices are \(0, \ a, \ ia, \ a+ia \). Now, we have two cases: the norm of \(b'\) is less than that of \(a\), in which case that's a contradiction, and the other in which the norm of \(b'\) is greater than or equal to that of \(a\). However, when we subtract \(a+ia\) from \(b\), we end up in the portion of the square whose vertices are \(0 \ -ia \ -a, \ -a-ia \) which is closer to the origin than before, another contradiction. Thus, \(\ZZ[i]\) is a principal ideal domain.
\begin{problem}
Use unique factorization in \(\ZZ[i]\) to prove that every prime \(p \equiv 1 \mod 4\) is a sum of two squares.
\end{problem}
It's a well known fact that \(\ZZ_p^*\) has a generator, say \(g\). Using this fact, one can prove that \(g^{\frac{p-1}{2}} \equiv -1 \mod p\) using basic number theory. Since \(p \equiv 1 \mod 4\), we have that \( \frac{p-1}{4} \) is an integer, and thus that \(g^{\frac{p-1}{4}}\) is precisely that \(n\) such that \(n^2 \equiv -1 \mod p\).

Observe that \(p|n^2+1=(n+i)(n-i)\), so evidently, \(p\) cannot be irreducible due to unique factorization, so we can write \(p=ab, \ a,b \in \ZZ[i] \) where neither is a unit. Since \(N(p)=p^2=N(a)N(b)\), we can conclude the desired statement.
We can describe all irreducibles in \(\mathbb{Z}[i]\) as elements whose norm is \(p\) or \(p^2\), where in the second, \( p \equiv 3 \mod 4\).

\begin{problem}
Show that \(\ZZ[\zeta_3]\) is yet another UFD.
\end{problem}

Observe that in this case, we define the norm in the prototypical way, the product of the images of the number in all of the different embeddings, thus \(N(a+b\zeta_3)=(a+b\zeta_3)(a+b\zeta_3^2)\). By a similar argument, we can reduce our lattice on \(\mathbb{C}\) into a parallelogram whose vertices are \(a, a\zeta_3, 0, a+a\zeta_3\), and so the point \(b\) which we assumed was not on the lattice will fall either in a place whose norm is less than that of \(a\), which contradicts the normality, or, we can translate it into a region (which is opposite the original), which has it in an area where the norm is less than that of \(a\). Thus \(\ZZ[\zeta_3]\) is also a UFD. (A side note is that when \(a \in \ZZ[\zeta_3]\) is written as a sum of real and imaginary components, the norm is the sum of the square of each ones magnitude.

\begin{problem}
Let \(x,y,z\) be a solution to \(x^4+y^4=w^2\) with smallest possible \(w\). Then \(x^2,y^2,w\) is a primitive Pythagorean triple. Assuming that \(x\) is odd, we can write \(x^2 = m^2-n^2, \ y^2 =2mn, \ w=m^2+n^2\) with \(m,n\) being relatively prime positive.
\begin{enumerate}[label=(\alph*)]
    \item Show that \(x = r^2-s^2, \ n=2rs, \ m=r^2+s^2 \) with \(r,s\) relatively prime integers.
\end{enumerate}
\end{problem}
Since \(x^2+n^2=m^2\) we can write \(x\) as \(x=r^2-s^2\), with \(r,s\) relatively prime, of different parity. Without loss of generality, suppose \(r,m\) are not relatively prime, then we must have that \(s\) shares the same factor, a contradiction. Evidently \(y^2=4rsm\) must be a product of squares, so \(r,s,m\) is \(a^2,b^2,c^2\). We use the fact that \(m=r^2+s^2\) to conclude the that the minimality of \(w\) is false.
\begin{problem}
Suppose that \(\ZZ[\omega]\) is a UFD and \(\pi | x+y\omega\). Show that \(\pi\) does not divide any of the other factors on the left side of \((x+y)(x+y\omega)\cdots (x+y\omega^{p-1}=z^p\). Furthermore show that \(x+y\omega = u \alpha^p\), where \(u\) is a unit.
\end{problem}
Suppose \(\pi\) is an irreducible, \(\pi | x+y\omega, \ x+y\omega^k \implies \pi | y\omega-y\omega^k\). Then, since \(yp = y(1-\omega)(1-\omega^2) \dots \implies \pi | yp, z\). However, by the assumption that \(p \nmid y,z \implies \gcd(z,yp)=1\), this cannot be. Since it's not true in the case of irreducibles, it cannot be true in the case of reducibles. Evidently, the right hand side is a product of irreducibles raised to some \(p\)th power, and the right hand side has the property that if \(\pi | x+y\omega\) then \(\pi \nmid x+y\omega^k\), so \(\pi\) must be something to the \(p\)th power. As you consider all divisors of \(x+y\omega\), we get that this is always true. Of course, this is only true up to units, from which the conclusion follows naturally.
\begin{problem}
Dropping the assumption that \(\ZZ[\omega]\) is a UFD, show that the principal ideal \((x+y\omega)\) has no prime ideal factor with any others in the equation \((x+y)(x+y\omega)\cdots (x+y\omega^{p-1} = (z)^p\), and similarly to the previous one, that \((x+y\omega)=I^p\).
\end{problem}